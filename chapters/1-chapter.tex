% ----------------------------------------------------------
\chapter{Introdução}
% ----------------------------------------------------------

A contagem de polígonos em artefatos gráficos em filmes, jogos, pesquisas medicas e cientificas seguem em crescimento
vertiginosa e apesar do Hardware acompanhar este crescimento, existem claras restrições com relação a \gls{V-RAM} e a
\gls{RAM}. Com essas restrições em mente, são necessárias estruturas de dados que permitam o acesso rápido dessas
figuras, consultas e armazenamento.
A seleção e determinação de faces visíveis \gls{VSD}



% \nocite{NBR6023:2002}
% \nocite{NBR6027:2012}
% \nocite{NBR6028:2003}
% \nocite{NBR10520:2002}

% ----------------------------------------------------------
\section{Objetivos}
% ----------------------------------------------------------

O objetivo deste trabalho visa a busca rápida de estruturas geométricas.
Visando aplicações gráficas em tempo real.
Quando a câmera da aplicação precisa saber quais figuras geométricas precisam ser desenhadas
tendo apenas a informação da posição da câmera e das coordenadas do mundo, este artigo visa
o estudo de algoritmos para a solução deste tipo de problema.

% % ----------------------------------------------------------
% \subsection{Objetivo Geral}
% % ----------------------------------------------------------

% Descrição...

% % ----------------------------------------------------------
% \subsection{Objetivos Específicos}
% % ----------------------------------------------------------

% Descrição...